\begin{fullwidth}
    \section{FEIP: Appendices} \label{app:feip_appendices}
\end{fullwidth}

    Appendix material supporting the Front Ends Incentive Program.

    \subsection{DYDX Rewards versus the Fee-Share Model} \label{subapp:liquity}

            Our proposed incentives program is inspired by the Liquity \bhref{https://www.liquity.org/features/3rd-party-frontends}{program} for decentralizing front ends. The Liquity model has two standout components: (1) it pays front end operators in LQTY proportional to the volume pushed by each front end, similar to the Trader Rewards or LP Rewards modules established in dYdX v3, and (2) Front End operators can choose how much of the rewards are sent back to users, denoted as a ``kickback'' rate.
            
            A DYDX-denominated rewards program has advantages and disadvantages. A DYDX rewards program is paid for by the community treasury and tokenholders, whereas the fee revenue share is paid for by validators and stakers. By burdening the community treasury instead of validators, a DYDX-denominated rewards program might be a better choice for maintaining the POS security of the app chain. That is, higher fee revenue increases the incentives for staking DYDX to secure the chain and share in the upside. A DYDX-denominated rewards program would also incentivize the creation of front ends despite the relatively low volume that might be observed at the Genesis of dYdX chain. Given the importance of ensuring adequate validator incentives, this might be the appropriate initial funding scheme.
            
            However, as a long-term solution to decentralizing front ends, a DYDX rewards program poses several operational concerns. First and foremost, it relies on the limited supply of DYDX token, discussed in Section \ref{subsec:emissions}. As we approach the terminal inflation rate for DYDX, the community will have to make several difficult decisions on which incentives programs and development efforts to prioritize. Certain programs, such as incentivizing front ends or other infrastructure components of dYdX v4, might be better funded via a fee share, allowing the community to leverage the remaining DYDX token on other discretionary efforts.

            Furthermore, DYDX incentives depend entirely on the price of DYDX. As the price of DYDX fluctuates, the incentives to develop and maintain front ends will fluctuate with it. To keep such a program effective, the DAO would have to rebalance the program's DYDX allocation to prevent it from over or under spending. If DYDX price increases materially from the moment the program is originally sized, the program would unnecessarily drain the community treasury. If price decreases, the program quickly becomes ineffective. Xenophon Labs raised a similar efficiency concern in sizing the safety module in a recent publication \cite{holloway2023dydx}. We would avoid a dependency on DYDX price by incentivizing front end operators in USDC-denominated fee sharing, aligning the front end operator's revenue with their costs, such as software engineers, or AWS subscriptions. Instead, the program would then become dependent on trading volume, which aligns the front end operator's incentives with the protocol's incentives.

            Finally, DYDX rewards create opportunities for users to ``game'' the incentives program. This played a significant role in how and why traders farmed the DYDX token during the earlier days of the dYdX Trader rewards program, as we discussed in our 2022 review of the Trader Rewards Module \cite{cintra-holloway}. This might lead front end operators to trade through their own platforms to take a larger share of the DYDX rewards. This has resulted in several governance proposals to move away from DYDX rewards programs, such as winding down the Trader rewards module, or move away from DYDX LP rewards and towards LP rebates. In general, DYDX rewards can be gamed when users find an opportunity to earn more in DYDX rewards than they pay in fees to earn those rewards. A fee share avoids this problem entirely.

            % While our proposal doesn't explicitly cover a kickback rate, nothing prevents front end operators from providing such a service to their users. This might be one way in which operators increase their share of the front end market: by returning some of their incentives to their users. With a kickback rate at 100\% and no whitelisting process, Liquity's front end incentives program is essentially the same as trading rewards: pay users to take out Liquity loans. Most of Liquity's front ends offer a kickback rate of $98\%+$. 


        \subsection{FEIP: Sizing}
            
            As dYdX mainnet launches, it is possible that various front ends are created in many jurisdictions around the world with no incentives required. In this case, a front ends incentives program is unnecessary. An example of this is Liquity's decentralized front ends, many of which return $100\%$ of their rewards to their users. That is, there are external incentives for deploying front ends outside of receiving a share of trading fees, and these are sufficient for many front ends to be deployed and maintained. While this foregoes the objective of a competitive market that encouraged innovation and growth, trading fees and developer resources might be better spent on other aspects of the protocol. Liquity's design is discussed in Appendix \ref{subapp:liquity}.
    
            In the event that there are an insufficient amount of front ends being deployed following the launch of v4, or there is appetite from the community to fund innovation and growth with respect to dYdX's user interface, we might launch a front end incentives program with a small fee-share allocation. We suggest the following scheme for initially setting the fee share percentage: (1) estimate the running annual costs for maintaining a front end, (2) record the recent annualized trading fees generated from front end users, (3) size a fee-share such that, if incentives were equally distributed amongst participants, $N$ front ends are able to operate profitably. 
    
            In order to discover the true running costs for these front end operators, we might start with an underestimate for these costs, and ratchet these up until enough front end operators have applied for the incentives program, and deployed their platform. For example, we might estimate their annual running costs, meaning hardware costs and a part time engineer to maintain the codebase, is around $\$50k$ USD. Suppose that the recent $\$2.7M$ in per-epoch trading fees from front ends continues to be true in v4, and that the community is targeting to incentivize $10$ front ends. Then, we might initially size the fee-share percentage at $\approx 1.5\%$. If $\$50k$ is not a reasonable reflection of running costs, few teams might apply for the incentives program, and we might then propose to raise the fee share percentage. As dYdX v4 gains more traction and trading volume increases, this fee-share percentage might then be revised and reduced.
    
            It is possible that governance will want to prioritize the decentralization of the front end, but there is insufficient revenue being generated through trading fees to fund this program without significantly damaging validator profits. In this case, the community treasury might be used to fund the incentives program for a limited time with the goal of increasing trading via front ends. Once a critical amount in trading fees is achieved, we may then switch to the proposed trading fee approach. 
