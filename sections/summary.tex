\begin{fullwidth}
    
    \begin{adjustwidth}{1cm}{1cm}
        \section{Executive Summary} \label{sec:summary}
    
        \textcolor{gray}{\rule{\linewidth}{0.1mm}}
    
        dYdX v4 marks dYdX's transition to a fully decentralized exchange, owned and operated by the dYdX community as a standalone Cosmos blockchain. The dYdX Chain, as its called, features various improvements with respect to decentralization, scalability, and customizability. With its decentralized, off-chain order book and matching engine, dYdX Chain will enable significantly higher transaction throughput without the need for oversight from a centralized 3rd party. Furthermore, being its own Cosmos blockchain, dYdX Chain also empowers the dYdX community to customize several key properties of the protocol. This includes punishing, and perhaps even preventing the extraction of MEV (that is, Maximal Extractable Value), and ensuring dYdX Chain traders continue to pay trading fees, but not gas fees, as they do on dYdX v3.

        dYdX v4 has been in the works for well over a year, with a planned mainnet launch in Q4 of 2023. In anticipation of this launch, the authors have written this report in hopes of familiarizing the reader with the unique challenges that dYdX governance has faced in migrating from Ethereum to Cosmos, how it has overcome those challenges, and how it might tackle the challenges that lie ahead. This report is split into two parts.

        \subsection*{Part 1: Recapping dYdX’s Migration from Ethereum to Cosmos} 
        
        We begin with a brief overview of dYdX v4, followed by a chronology of the various steps taken by the dYdX Community to migrate the dYdX ecosystem from Ethereum to Cosmos. This includes:
        
        \begin{itemize}
            \item Adopting the dYdX v4 open-source software developed by dYdX Trading as the next version of the dYdX protocol.
            \item Establishing DYDX as the L1 token for dYdX Chain.
            \item Winding down dYdX v3 ecosystem incentives.
            \item Bridging community resources to dYdX Chain.
            \item Deploying novel incentives programs to accelerate the adoption of dYdX v4.
        \end{itemize}

        As of early October 2023, some of these are still a work in progress. 
    
        \subsection*{Part 2: The Challenges that Lie Ahead} 
        
        Next, this report overviews the many critical components of dYdX v4 which the community will be largely responsible for researching, developing, and maintaining. This includes: % \sidenotequote{With v4 nearing and dYdX’s subsequent decentralization, a major responsibility is put on stakeholders and the community to ultimately govern and grow dYdX into its hegemony over decentralized derivatives.}{\bhref{https://dydx.forum/t/v4-vanguard-thoughts-about-dydxs-future}{v4 Vanguard - Thoughts About dYdX’s Future}}
    
        \begin{itemize}
            \item Monitoring MEV activity and slashing misbehaving validators.
            \item Managing trading fee tiers and rebates.
            \item Monitoring and adjusting the Trading Rewards program.
            \item Listing new markets via permissioned or permissionless listings.
            \item Managing market risk parameters and monitoring missed liquidations.
            \item Implementing novel incentives programs targeting key behaviors across the dYdX Chain ecosystem.
            \item Managing governance proposals, subDAOs, and other governance parameters.
        \end{itemize}
    
        This report also includes suggestions from Xenophon Labs and other community members on how the dYdX Community might tackle some of the challenges pertaining to dYdX v4. Many of these suggestions have been posted on the community's forum, and we will be referencing them throughout the report. 
    
        \subsection*{A Note to the Reader} 
        
        This is a living document with versions documented in the changelog. It lives in \bhref{https://github.com/xenophonlabs/dYdX-v4-Report}{this} github repository; contributors are welcome and encouraged.
    
        \begin{center}
            \large
            This report documents dYdX's migration from v3 to v4, and the role of the dYdX Community in operating dYdX Chain. It is for informational purposes only. Thank you for reading.
        \end{center}
    
    \end{adjustwidth}

\end{fullwidth}